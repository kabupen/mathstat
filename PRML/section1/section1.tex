\documentclass[10pt, a4paper]{ltjsarticle}

\usepackage{../../mathstat}

\begin{document}

\section{多項式曲線フィッテイング}

回帰問題では観測値 $(\bm{x}, \bm{t})$ の組み合わせを用いて、未知の入力変数の値に対して目的変数の値を
予測する問題になる。有限事のデータ集合から汎化しなければならない。
データに対する予測値を $y(x,w)$ とすると、線形モデル\footnote{データ$x$に対しては非線形モデルであるが、パラメータ$w$については線形モデルである}では
\begin{equation}
  y(x, \bm{w}) = w_0 + w_1x + w_2x^2 + ... = \sum w_jx^j
\end{equation}
予測値$y$と目的変数$t$(どちらも観測値の情報)のズレを誤差関数として表し
\begin{equation}
  E(w) = \frac{1}{2}\sum \{y(x_n, w) - t_n\} ^2
\end{equation}
この誤差が最小となるようにパラメータ $w$ を選ぶことで、回帰問題を解く。


\section{多項式曲線フィッテイング(その2)}

曲線フィッテイングでは、実際には観測値$x$に対して目的変数 $t$ はノイズが乗っていると考えられ、不確実性がある。
この不確実性を表現するために確率分布を用いることができる。入力情報 $x$ に対して、対応する$t$ は平均が $y(x,w)$、
分散が $\beta^{-1}$ で与えられるガウス分布に従うとすると
\begin{equation}
  p(t|x,w,\beta) = N(t|y(x,w), \beta^{-1})
\end{equation}
となる。
こうすることで、訓練データ $(\bm{x},\bm{t})$ を用いることで、最尤法からパラメータ $w,\beta$ を求めることができる。

\end{document}
