\documentclass[10pt, a4paper]{ltjsarticle}

\usepackage{../../mathstat}

\begin{document}

\section{ベクトル微分}

ベクトルで微分するときにも、要素ごとに微分するという考え方で微分計算をすればよく、深く考え込みすぎなくてよい。

\subsection{vector-by-scalar}

\begin{equation}
  \frac{\partial \bm{y}}{\partial x} = \left(\frac{\partial y_1}{\partial x}, \frac{\partial y_2}{\partial x},..., \frac{\partial y_m}{\partial x} \right)
\end{equation}


\subsection{scalar-by-vector}

\begin{equation}
  \frac{\partial y}{\partial \bm{x}} = \left(\frac{\partial y}{\partial x_1}, \frac{\partial y}{\partial x_2},..., \frac{\partial y}{\partial x_n} \right)
\end{equation}

勾配ベクトル
\begin{equation}
  \nabla f(x,y,z) = \left(\frac{\partial f}{\partial x}, \frac{\partial f}{\partial y}, \frac{\partial f}{\partial z} \right)
\end{equation}

\subsection{vector-by-vector}

\begin{equation}
  \frac{\partial \bm{y}}{\partial \bm{x}} = 
  \begin{bmatrix}
    \frac{\partial y_1}{\partial x_1} & \frac{\partial y_1}{\partial x_2} & \hdots &\frac{\partial y_1}{\partial x_n}  \\
    \vdots \\
    \frac{\partial y_m}{\partial x_1} & \frac{\partial y_m}{\partial x_2} & \hdots &\frac{\partial y_m}{\partial x_n}  \\
  \end{bmatrix}
\end{equation}






\subsection{ヘッセ行列}
ヘッセ行列(hessian matrix)は
\begin{equation}
  H_{ij}  = \frac{\partial^2}{\partial x_i\partial x_j} f(\bm{x})
\end{equation}
である。

\end{document}
