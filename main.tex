\documentclass[10pt,a4paper]{jsarticle}

\usepackage{amsmath,amssymb}
\usepackage{bm}
\usepackage{graphicx}
\usepackage{ascmac}

\setlength{\textwidth}{\fullwidth}
\setlength{\textheight}{39\baselineskip}
\addtolength{\textheight}{\topskip}
\setlength{\voffset}{-0.5in}
\setlength{\headsep}{0.3in}

\begin{document}

\section{例1.5}

\begin{equation}
    _nC_{m-1}x^{m-1}(1-x)^{n-m+1}\times _{n-m+1}C_1\frac{dx}{1-x}\left(1-\frac{dx}{1-x}\right)^{n-m}
\end{equation}

ここで $dx$ が微小量であることを用いて、最後の部分をテイラー展開する\footnote{2022/05/19:この展開公式を忘れていた。Taylor展開は頭に入れ直しておこう...。}。
$dx$の二次の項は微小量であるとして、
\begin{equation}
  \left(1-\frac{dx}{1-x}\right)^{n-m} \simeq 1-(n-m)\frac{dx}{1-x} + \mathcal{O}\left(dx^2\right)
\end{equation}
以上より
\begin{equation}
\frac{dx}{1-x}\left(1-\frac{dx}{1-x}\right)^{n-m} \simeq \frac{dx}{1-x} \times \left( 1-(n-m)\frac{dx}{1-x} \right) \simeq \frac{dx}{1-x} + \mathcal{O}\left(dx^2\right)
\end{equation}

\section{条件付き確率}
\section{中心極限定理}
\section{大数の法則}

\end{document}
