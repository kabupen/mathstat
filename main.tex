\documentclass[10pt,a4paper]{ltjsarticle}

\usepackage{amsmath,amssymb}
\usepackage{bm}
\usepackage{graphicx}
\usepackage{ascmac}

\setlength{\textwidth}{\fullwidth}
\setlength{\textheight}{39\baselineskip}
\addtolength{\textheight}{\topskip}
\setlength{\voffset}{-0.5in}
\setlength{\headsep}{0.3in}

\begin{document}


\section{例1.5}

\begin{equation}
    _nC_{m-1}x^{m-1}(1-x)^{n-m+1}\times _{n-m+1}C_1\frac{dx}{1-x}\left(1-\frac{dx}{1-x}\right)^{n-m}
\end{equation}

ここで $dx$ が微小量であることを用いて、最後の部分をテイラー展開する\footnote{2022/05/19:この展開公式を忘れていた。Taylor展開は頭に入れ直しておこう...。}。
$dx$の二次の項は微小量であるとして、
\begin{equation}
  \left(1-\frac{dx}{1-x}\right)^{n-m} \simeq 1-(n-m)\frac{dx}{1-x} + \mathcal{O}\left(dx^2\right)
\end{equation}
以上より
\begin{equation}
\frac{dx}{1-x}\left(1-\frac{dx}{1-x}\right)^{n-m} \simeq \frac{dx}{1-x} \times \left( 1-(n-m)\frac{dx}{1-x} \right) \simeq \frac{dx}{1-x} + \mathcal{O}\left(dx^2\right)
\end{equation}

\section{式(1.30)}

ポワソン分布のモーメント母関数は

\begin{equation}
  M(t) = E[e^{tX}] = \sum_{k=0}^{\infty} e^{tk}\times \frac{e^{-\lambda}\lambda^k}{k!} = e^{-\lambda }\sum_{k=0}^{\infty} \frac{e^{tk}\lambda^k}{k!}
\end{equation}

ここで $e^{ax}$ のテイラー展開が

\begin{equation}
  e^{ax} = \sum \frac{a^k}{k!} x^k
\end{equation}

であることを使用して、$a=e^t$と$x=\lambda$を代入すると

\begin{equation}
  M(t) = E[e^{tX}] = exp[-\lambda(1-e^t)]
\end{equation}

となる。

\section{条件付き確率}

\section{中心極限定理}


\section{中心極限定理}

\section{ガウスマルコフの定理}

線形モデル $y = X\theta + \epsilon$ に関する任意の推定可能関数 $\ell^T\theta$ について、$\ell^T\hat{\theta}$ が BLUEを与える。
ただし $\hat{\theta}$ は正規方程式 $X^TX\hat{\theta} = X^T\theta$を満たす。


\section{確率分布}

\subsection{カイ二乗分布}
\subsection{t分布}
\subsection{F分布}


\section{線形代数}


\begin{equation}
  \frac{\partial A^Tx}{\partial x} = \frac{\partial x^TA}{\partial x}= A
\end{equation}

\begin{itemize}
  \item 行列$A$がフルランク→行列$A$は正則(逆行列を持つ)
\end{itemize}

\section{収束}

\subsection{確率収束}

確率変数列 $X_n$ が確率変数 $X$ に確率収束するとは、任意の $\epsilon > 0$ に対して、
\begin{equation}
  \lim_{n\to\infty} P(|X_n - X|>\epsilon) = 0
\end{equation}
が成り立つことをいう。

\subsection{分布収束}

\section{統計的推定}

\section{不偏性}

推定量が不偏推定量であるとは、すべての$\theta$に対して
\begin{equation}
  E[\hat{\theta}] = \theta
\end{equation}
が成り立つことをいう。



\section{フィッシャー情報量}



\section{検定論}

標本 $X$ から判断を決定 $d$ するための関数を決定関数 $\delta$ といい、検定論では「検定関数」という。

\subsection{用語}

検定の標準的な損失関数として決定関数を用いて

\subsection{用語}

\begin{itemize}
  \item 決定:データから判断(ex. 帰無仮説をどうするかなど...)すること
  \item 損失関数:決定の結果被る損失を数式で評価しているもの
  \item リスク関数:損失関数の期待値
  \item 決定関数(検定関数)
\end{itemize}


\subsubsection{過誤}

帰無仮説 $H_0$ が正しいときに帰無仮説を棄却する誤りを第一種の過誤、
帰無仮説 $H_0$ が正しくないのに帰無仮説を受容する誤り(対立仮説 $H_1$ が正しいときに帰無仮説を受容する誤り)を第二種の過誤という。
損失関数を用いると
\begin{equation}
  L(\theta, 0)
  = \left\{
    \begin{array}{ll}
    0,~~\rm{if~~\theta \in \Theta_0} \\
    1,~~\rm{if~~\theta \in \Theta_1} 
    \end{array}
  \right.
\end{equation}
\begin{equation}
  L(\theta, 1)
  = \left\{
    \begin{array}{ll}
    1,~~\rm{if~~\theta \in \Theta_0} \\
    0,~~\rm{if~~\theta \in \Theta_1} 
    \end{array}
  \right.
\end{equation}
と表すことができる。

ここで、帰無仮説 $H_0$ が正しいと判断することを「受容する」、帰無仮説が間違っていると判断することを「棄却する」という。
これらの決定をそれぞれ $d=0$、$d=1$ と表す。


\subsection{検出力(power)}

\begin{itembox}[l]{検出力関数(power function)}
  決定が $d=1$ となる確率を検出力といい、その値を計算するための関数を
\begin{equation}
  \beta_\delta(\theta) = P_\theta(\delta(X)=1) = P_\theta(X\in R)
\end{equation}
として定義し、検出力関数という。
\end{itembox}

検出力関数では$\theta$が$\Theta_0,\Theta_1$のどちらに属しているか(真にどちらが正しいか)には言及していない。
単に検定関数 $\delta(x)$ が 1 になる、つまり帰無仮説を棄却する確率を表しているに過ぎない。


\subsection{受容/棄却域}

帰無仮説 $H_0$ が受容されるか棄却されるかの観点で、標本空間を分けることができる。$A=\{x|\delta(x)=0\}$ を受容域、$A^c=R=\{x|\delta(x)=1\}$ を棄却域と呼ぶ。



\end{document}
