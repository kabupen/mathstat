\documentclass[10pt, a4paper]{ltjsarticle}

\usepackage{amsmath,amssymb}
\usepackage{listings}

\usepackage{minted}
\usepackage{xcolor}
\definecolor{lightgray}{gray}{0.95}

\usepackage{graphicx}
% \usepackage{bm}
% \usepackage[pdftex]{graphicx}
% \usepackage{ascmac}

% \setlength{\textheight}{39\baselineskip}
% \addtolength{\textheight}{\topskip}
% \setlength{\voffset}{-0.5in}
% \setlength{\headsep}{0.3in}

\usepackage{bookmark}
\usepackage{xurl}
\hypersetup{
  unicode,
  bookmarksnumbered=true,
  hidelinks,
  colorlinks=true,
  citecolor=blue,
  linkcolor=blue,
  urlcolor=blue,
  final}

\newcommand\refeq[1]{式(\ref{#1})}

\begin{document}

\section{多項分布の一様性検定}

\subsection{一様性検定}

帰無仮説 $H_0: p_{i1},p_{i2},p_{i3}=p_1,p_2,p_3$の検定として、一様性を検定した。
ただし不均一性が議論できたとして、そこから一歩進めてどのような不均一性か、どのようなモデルが適合するのか、を議論する必要がある。


%~~~~~~~~~~~~~~~~~~~~~~~~~~~~~~~~~~~~~~~~~~%
\subsection{$\chi^2$統計量の分解}
%~~~~~~~~~~~~~~~~~~~~~~~~~~~~~~~~~~~~~~~~~~%

$a\times b$の分割表(cross table)において、(5.10)に示した帰無仮説 $H_0 : p_{ij} = p_i\cdot p_j$ のもとでの適合度検定は、
式(5.13)の様な総括的な$\chi^2$統計量を使用していた。
\begin{eqnarray}
  \chi^2 &=& \sum_{i=1}^a\sum_{j=1}^b  \frac{\left(y_{ij} - \hat{y_{ij}} \right)^2 }{\hat{y_{ij}}} \\
  &=& \sum_{i=1}^a\sum_{j=1}^b \cfrac{\left(y_{ij} - \cfrac{y_{i\cdot}y_{\cdot j}}{n} \right)^2}{\cfrac{y_{i\cdot}y_{\cdot j}}{n}}
\end{eqnarray}

この$\chi^2$統計量は群間の差を表す成分に分解することができる。$\{1,2,...,p+q\}$を2群$I_1=\{1,2,...,p\},~I_2=\{p+1,...,p+q\}$に分割するとき、その群間の差を表す$\chi^2$成分は
\begin{equation}
  \chi^2(I_1;I_2) = \cfrac{y_{\cdot\cdot}\sum\limits_{j=1}^b
  \left(
    \sum\limits_{l=1}^p y_{lj} / \sum\limits_{l=1}^p y_{l\cdot} 
    - \sum\limits_{l=p+1}^{p+q} y_{lj} / \sum\limits_{l=p+1}^{p+q} y_{l\cdot} 
    \right)^2/y_{\cdot j}
  }{
  1/\sum\limits_{l=1}^p y_{l\cdot} + 1/\sum\limits_{l=p+1}^{p+q} y_{l\cdot}  
  }
  \label{eq:5.20}
\end{equation}
で与えられる(式(5.20))。
これを用いて全体の$\chi^2$を
\begin{equation}
  \chi^2 = \chi^2(1;2) + \chi^2(1,2;3) + ... + \chi^2(1,...,p;p+1,...,p+q)+...+\chi^2(1,...,a-1;a)
\end{equation}
と分解することができる。式中盤に出てくる$\chi^2(1,...,p;p+1,...,p+q)$は\refeq{eq:5.20}に相当する。$a=p+q=2$の場合を考えると全体の$\chi^2$は
\begin{eqnarray}
  \chi^2 &=& \chi^2(1;2) + \chi^2(1,2; 3) \\
  &=& \chi^2(1;3) + \chi^2(1,3;2) \\
  &=& \chi^2(2;3) + \chi^2(2,3;1)
\end{eqnarray}
のように、行の入れ替えを行うことで適宜群の定義を変えて計算を行うことができる。

これは自由度 $\nu=(a-1)(b-1)$のカイ二乗分布に近似的に従う。

\begin{eqnarray}
  \chi^2 = \sum_{i=1}^{p+q}\sum_{j=1}^b \cfrac{\left(y_{ij} - \cfrac{y_{i\cdot}y_{\cdot j}}{n} \right)^2}{\cfrac{y_{i\cdot}y_{\cdot j}}{n}}
\end{eqnarray}

全体の $\chi^2$は、$I_1=\{1,2,...,a-1\},~I_2=\{a\}$、さらに$I_1$に対して同様の分割を繰り返すことで式(5.21)が成り立つ。
\begin{eqnarray}
  \chi^2 &=& \chi^2(I_1; I_2) \\ 
\end{eqnarray} 



% =================================== %
\subsection{順序カテゴリのモデル化}
% =================================== %




\newpage

\section{分割表の対称性の検定}


行と列が同種の属性を表す正方な分割表(cross table)にまとめられるデータは、基本的に対角成分が多くなる(=変化がない)。
そのため通常の独立性の検定には意味がない。


独立性の検定とは $p_{ij} = p_ip_j$ を仮定する独立モデルの適合度検定である。



\subsection{対称性}

行と列が同じ分類からなる正方分割データ\footnote{患者の治療の前後の病状、左右の裸眼の視力など~\cite{hirotsu}}について考える。
このようなデータの場合、対角線上にデータが集中するため属性間の独立が成り立たないことが多い(=前節までの独立性の検定は意味がない)。このようなデータに対して興味があるのは、対角線に関して対称な位置にある成分の生起確率が同じであるかどうかである。

教科書表5.11について、支持政党の推移が無いとする帰無仮説は
\begin{equation}
  H_0 : p_{ij} = p_{ji}
\end{equation}
で表される。これはある政党だけ支援者数が増加する傾向があるなどといった構造がないことを意味していて、全体として支持政党の推移はバランスが取れていることを仮定している(対称性がある)。
対称性の検定では帰無仮説の構造についての検定と
そこからのズレを議論する\footnote{対称性の解析には様々なモデルがあるそうで、Bowkerの対称モデル、Caussinusの準対称モデル、Stuartの周辺対称モデル、条件付き対称モデル、対角パラメータ対称モデル、線形対角パラメータ対称モデル、などなど。。。}。
これは正方な分割表における対称性の適合度検定と呼ばれている。

多項分布で考えるとデータが得られる確率(尤度関数)は式(5.4)を参考にして(セルっぽく$i$と$j$を使って)
\begin{equation}
  L(p) = \cfrac{n!}{\prod_i\prod_j y_{ij}!} \prod_i\prod_j p_{ij}^{y_{ij}} = n!\prod_i\prod_j \cfrac{p_{ij}^{y_{ij}}}{y_{ij}!}
\end{equation}
と書ける。帰無仮説$H_0:p_{ij}=p_{ji}$のもとで、$\sum_{i,j}p_{ij}=1$の束縛条件のもとでラグランジュの未定乗数法を使って
\begin{eqnarray}
  K = \log L -\lambda \left(\sum p_{ij}-1 \right)
\end{eqnarray}
を最適化する。ここで
\begin{eqnarray}
  \log L = \log n! + \sum_i\sum_j\left(y_{ij}\log p_{ij} - \log y_{ij}! \right) 
\end{eqnarray}
である。
$i=j$、$i\neq j$の場合で偏微分の結果が変わることを考慮して
\begin{eqnarray}
  \cfrac{\partial K}{\partial p_{ii}} &=& \frac{y_{ii}}{p_{ii}} - \lambda = 0 \label{eq:5.26-1}\\
  \cfrac{\partial K}{\partial p_{ij}} &=& \frac{y_{ij}}{p_{ij}} + \frac{y_{ji}}{p_{ji}} - 2\lambda = \frac{y_{ij}+y_{ji}}{p_{ij}} - 2\lambda = 0 \label{eq:5.26-2}\\
  \cfrac{\partial K}{\partial \lambda} &=& \sum\sum p_{ij} = 1
\end{eqnarray}
最後の式に代入して
\begin{eqnarray}
  \sum_i\sum_j \cfrac{y_{ij}}{\lambda} + \sum_i\sum_j \cfrac{y_{ij}+y_{ji}}{2\lambda} &=& 1 \\
  \therefore \lambda &=& n  
\end{eqnarray}
ゆえに最尤推定量を求めると、\refeq{eq:5.26-1}と\refeq{eq:5.26-2}から式(5.26)が得られる。

また、$(i,j)$ のセルの生起確率は
\begin{eqnarray}
  \tilde{p_{ij}} = \frac{y_{ij}+y_{ji}}{2n}
\end{eqnarray}
とまとめて表せて($i=j$の場合は式(5.26)の左側に帰着する)、これを用いると適合度検定の統計量が得られる。
\begin{eqnarray}
  \chi^2 &=& \sum_i\sum_j\cfrac{\left(y_{ij} - n\tilde{p_{ij}}\right)^2}{n\tilde{p_{ij}}} \\
  &=& \sum_i\sum_j \left( y_{ij} - \cfrac{y_{ij}+y_{ji}}{2} \right)^2/\left(\cfrac{y_{ij}+y_{ji}}{2}\right) \\ 
  &=& \mathop{\sum_i\sum_j}_{i\neq j} \left( \frac{1}{2} (y_{ij} - y_{ji}) \right)^2/\left(\cfrac{y_{ij}+y_{ji}}{2}\right) \\ 
  &=& 2 \times \mathop{\sum_i\sum_j}_{i < j} \left( \frac{1}{2} (y_{ij} - y_{ji}) \right)^2/\left(\cfrac{y_{ij}+y_{ji}}{2}\right) \\ 
  &=& \mathop{\sum_i\sum_j}_{i < j} \left( y_{ij} - y_{ji} \right)^2/\left(y_{ij}+y_{ji}\right) 
\end{eqnarray}
2行目から3行目には、$i=j$の場合に分母がゼロになることを、3行目から4行目には$i,j$の対称性から$i<j$の条件をつけた和で表現した。以上が式(5.27)の導出過程の理解である。





\subsection{周辺対称性と準対称性}

対称性の仮説は周辺対称性と準対称性の2つの仮説に分解される\footnote{Caussinusは対称モデルの分解定理?、「対称モデルが成り立つための必要十分条件は、周辺対称モデルと準対称モデルの両方が成り立つことである」ことを示した}。

\subsubsection{周辺対称性}

周辺確率が等しい、という仮説。

\subsubsection{準対称性}

\newpage

\section{ブラッドリーテリーのモデル}

- \url{https://www.gavo.t.u-tokyo.ac.jp/~mine/japanese/IT/2017/toukei171211.pdf}

強さをモデル化して議論するためのモデルを、ブラッドリー・テリーのモデルという。
強さを表すパラメータを$\theta_i$としたときに、$i$が$j$に勝つ確率を$p_{ij}$として
\begin{equation}
  p_{ij} = \frac{\theta_i}{\theta_i + \theta_j}
\end{equation}
で表す。このパラメータは全体を定数倍しても確率値には影響しない。
\begin{equation}
  p_{ij} = \frac{c\theta_i}{c\theta_i + c\theta_j} =  \frac{\theta_i}{\theta_i + \theta_j}
\end{equation}

ブラッドリー・テリーのモデルは「苦手」を表すことはできず(ex. じゃんけんのようにグーはパーが苦手=勝てない、などは表現できない)、
三すくみの関係性がないモデルと解釈することができる。$i$が$j$に、$j$が$k$に、$k$が$i$に勝つ確率を$p_{ij}, p_{jk}, p_{ki}$としたときに
\begin{equation}
  p_{ij}p_{jk}p_{ki} = \frac{\theta_i\theta_j\theta_k}{(\theta_i + \theta_j)+(\theta_j+\theta_k)(\theta_k+\theta_i)} = p_{ji}p_{kj}p_{ik}
  \label{eq:5.34}
\end{equation}
となる。仮に三すくみの関係($i$は$j$によく勝つ、$j$は$k$によく勝つ、$k$は$i$によく勝つ)があったとすると、
\begin{equation}
  p_{ij}p_{jk}p_{ki} > p_{ji}p_{kj}p_{ik}
\end{equation}
となることが分かる。\refeq{eq:5.34}からその三すくみの関係性を仮定していないのがブラッドリー・テリーのモデルであることが分かる。

%~~~~~~~~~~~~~~~~~~~~~~~~~~~~~~~~~~~~~~~~~~%
\subsection{式(5.35)について}
%~~~~~~~~~~~~~~~~~~~~~~~~~~~~~~~~~~~~~~~~~~%
式(5.32)はデータが生成される確率であり尤度関数\footnote{bishop上p.22、尤度関数はパラメータベクトルを固定位したときに観測された
データ集合がどれくらい起こりやすいかを表しているもので、$p(D|w)$と表される。}として考えることができるので、その対数を取ると
\begin{eqnarray}
  \log L &=& \log\mathop{\prod\prod}_{1\leq i \leq j \leq 6} {}_{y_{ij}+y_{ji}}C_{y_{ij}}(p_{ij})^{y_{ij}}(p_{ji})^{y_{ij}} \\
  &=& \mathop{\sum\sum}_{1\leq i \leq j \leq 6} \log {}_{y_{ij}+y_{ji}}C_{y_{ij}}(p_{ij})^{y_{ij}}(p_{ji})^{y_{ji}} \\
  &=& \mathop{\sum\sum}_{1\leq i \leq j \leq 6} 
    \left\{ \log({}_{y_{ij}+y_{ji}}C_{y_{ij}}) + y_{ij}\log p_{ij} + y_{ji}\log p_{ji} \right\} \\
  &=& \mathop{\sum\sum}_{1\leq i \leq j \leq 6} 
    \left\{ \log({}_{y_{ij}+y_{ji}}C_{y_{ij}}) + y_{ij}\log \cfrac{\theta_i}{\theta_i+\theta_j} + y_{ji}\log \cfrac{\theta_j}{\theta_i+\theta_j} \right\} \\
  &=& \mathop{\sum\sum}_{1\leq i \leq j \leq 6} 
    \left\{ \log({}_{y_{ij}+y_{ji}}C_{y_{ij}}) + y_{ij}(\log\theta_i - \log(\theta_i+\theta_j)) + y_{ji}(\log\theta_j - \log(\theta_i+\theta_j)) \right\} \\
  &=& \mathop{\sum\sum}_{1\leq i \leq j \leq 6} 
    \left\{ \log({}_{y_{ij}+y_{ji}}C_{y_{ij}}) -(y_{ij}+y_{ji})\log(\theta_i+\theta_j)) + y_{ij}\log\theta_i + y_{ji}\log\theta_j \right\} \\
  &=& \mathop{\sum\sum}_{1\leq i \leq j \leq 6} 
    \left\{ \log({}_{y_{ij}+y_{ji}}C_{y_{ij}}) -(y_{ij}+y_{ji})\log(\theta_i+\theta_j))\right\} + \sum_i\sum_j(y_{ij}\log\theta_i + y_{ji}\log\theta_j ) \label{eq:5.35}
\end{eqnarray}
となる。教科書では$r_{ij}=y_{ij}+y_{ji}$、$y_{i\cdot}=\sum_j y_{ij}$を用いている。

%~~~~~~~~~~~~~~~~~~~~~~~~~~~~~~~~~~~~~~~~~~%
\subsection{式(5.36)について}
%~~~~~~~~~~~~~~~~~~~~~~~~~~~~~~~~~~~~~~~~~~%
計算した対数尤度\refeq{eq:5.35}と、制約
\begin{equation}
  \sum_{i=1}^a \theta_i = K \label{eq:5.37}
\end{equation}
の元でラグランジュの未定乗数法を使って最尤推定量を求める。
ただし$i=j$の場合、対数の中身が1/2になって偏微分がゼロになるため、$i \neq j$の場合にのみに絞って考える\footnote{\url{https://personal.psu.edu/drh20/papers/bt.pdf}によると、「$w_{ij}$ denotes the number of times individual $i$ has beaten individual $j$ and we assume $w_{ii} = 0$ by convention」とあるので、教科書でも$y_{ii}=0$とするのが良いかもしれない。いずれにせよ$i\neq j$の場合のみを考える。}。
\begin{eqnarray}
  \mathcal{L} &=& \mathop{\sum\sum}_{1\leq i \leq j \leq 6} 
    \left\{ \log({}_{y_{ij}+y_{ji}}C_{y_{ij}}) -(y_{ij}+y_{ji})\log(\theta_i+\theta_j)) + y_{ij}\log\theta_i + y_{ji}\log\theta_j \right\} - \lambda \left(\sum_i\theta_i -K\right)
\end{eqnarray}
\begin{eqnarray}
  \frac{\partial}{\partial\theta_i} \log L &=& \mathop{\sum_{j}}_{j\neq i} \left( \frac{y_{ij}}{\theta_i} - \frac{y_{ij}+y_{ji}}{\theta_i + \theta_j} \right) - \lambda = 0 \label{eq:5.36-1}\\
  \frac{\partial}{\partial\lambda} \log L &=& \sum_i\theta_i -K = 0 \label{eq:5.36-2}
\end{eqnarray}
\refeq{eq:5.36-1}より(正直あやしいです)
\begin{eqnarray}
  \sum_j y_{ij} &=& \sum_j\frac{\theta_i}{\theta_i + \theta_j}(y_{ij}+y_{ji}) - \lambda\theta_i \\
  &=& \sum_j p_{ij}(y_{ij}+y_{ji}) - \lambda\theta_i \\
  &=& \sum_j \cfrac{y_{ij}}{y_{ij}+y_{ji}}\cdot(y_{ij}+y_{ji}) - \lambda\theta_i \\
  &=& \sum_j y_{ij} - \lambda\theta_i
\end{eqnarray}
ここでパラメータ$\theta_i\neq 0$であるので、$\lambda=0$。
よって最尤推定量は\refeq{eq:5.36-1}より
\begin{equation}
  \hat{\theta_i}\sum_j\frac{y_{ij}+y_{ji}}{\hat{\theta_i} + \hat{\theta_j}} = \sum_j y_{ij} \label{eq:5.36-2}
\end{equation}
となる。


%~~~~~~~~~~~~~~~~~~~~~~~~~~~~~~~~~~~~~~~~~~%
\subsection{逐次近似的について}
%~~~~~~~~~~~~~~~~~~~~~~~~~~~~~~~~~~~~~~~~~~%

式(5.36)のような形で最尤推定値が与えられているので、陽には解けず数値的に解く必要がある。初期値を(教科書ではある段階での
近似値)を$\hat{\theta_i^0}$ とする。\refeq{eq:5.36-2}を書き直して
\begin{equation}
  \hat{\theta_i}^1 =  \sum_j y_{ij} / \sum_j\frac{y_{ij}+y_{ji}}{\hat{\theta_i}^0 + \hat{\theta_j}^0}
\end{equation}
またこれは\refeq{eq:5.37}を満たすように適当にスケーリングして、
\begin{equation}
  \hat{\theta_i}^1 =  K \frac{\theta_i^1}{\sum\theta_i^1}
\end{equation}
こうすることである段階の値から次の段階の(???)最尤値が求まる。
これを繰り返すことで、
\begin{equation}
  \lim_{n\to\infty}\hat{\theta_i^n} = \hat{\theta_i}
\end{equation}
となることが知られている\footnote{よく分からず。Wikiには「This estimation procedure improves the log-likelihood in every iteration, and eventually converges to a unique maximum.」と書いているのみ。}。

教科書の「この手順を収束するまで繰り返す」というのが現実的にどの程度なのかよく分からないが、ある程度数値の変動がなくなったら
操作をとめるのか。


%~~~~~~~~~~~~~~~~~~~~~~~~~~~~~~~~~~~~~~~~~~%
\subsection{p.168ちょっとやってみた}
%~~~~~~~~~~~~~~~~~~~~~~~~~~~~~~~~~~~~~~~~~~%

\begin{minted}
[
frame=lines,
framesep=2mm,
baselinestretch=1.2,
bgcolor=lightgray,
fontsize=\footnotesize,
linenos
]{python}
y = np.array([[0, 14, 7, 13, 16, 18],
              [12, 0, 15, 8, 16, 17],
              [16, 9, 0, 12, 14, 12],
              [12, 17, 12, 0, 13, 7],
              [10, 10, 11, 12, 0, 11],
              [ 4, 8, 12, 16, 13, 0],
              ])
K = 100

np.random.seed(1)
theta = np.random.randn(6)

le_theta = np.zeros((6,100))

for itr in range(100):
    for i in range(len(theta)):
        A = 0.
        for j in range(len(theta)):
            A += (y[i,j]+y[j,i])/(theta[i]+theta[j])
        theta[i] = np.sum(y[i, :]) / A
    theta[i] = K * theta[i]/(np.sum(theta))
   
    for i in range(len(theta)):
        le_theta[i, itr] = theta[i]
\end{minted}

で逐次的に100回反復して求めると、次のような振る舞いになる。
20回くらいの思考でだいたい収束していることがわかる\footnote{スケーリングをどのタイミングでするのかは関係ない?実際コード21行目を for j のループ内に入れても特に変化なし}。

\begin{figure}[h]
 \centering
 \includegraphics[scale=0.6]{fig/brad}
 \caption{}
\end{figure}

求めた$\theta$を使って適合度検定を行うと、教科書p.169のようにブラッドテリーのモデルは
適していないことがわかる。
よって三すくみの関係性が実際にはあるということもわかる。


\newpage

%~~~~~~~~~~~~~~~~~~~~~~~~~~~~~~~~~~~~~~~~~~%
\section{3次元分割表と対数線形モデル}
%~~~~~~~~~~~~~~~~~~~~~~~~~~~~~~~~~~~~~~~~~~%

https://biolab.sakura.ne.jp/chi-square-residual-analysis.html


\newpage

\section{Appendix}

\subsection{$\chi^2$分布}

自由度$n$のカイ2乗分布は

\begin{equation}
  f(x) = \frac{1}{\Gamma(n/2)}\left(\frac{1}{2}\right)^{n/2}x^{n/2-1}\exp({-x/2})
\end{equation}

確率変数$Z$が標準正規分布に従うとき、$Z^2$が自由度1のカイ二乗分布($\chi^2_1$、$\chi^2(1)$)に従う。

カイ二乗分布も再生性を持っており、

\begin{equation}
  \chi^2_m + \chi^2_n = \chi^2_{m+n}
\end{equation}

が成り立つ。

$Z_1,...,Z_k$が互いに独立な確率変数とし、それぞれが標準正規分布に従うとする。このとき$Z_1,...,Z_k$の二乗和は自由度$k$のカイ二乗分布に従う
\begin{equation}
  Z_1^2+...+Z_k^2 \sim \chi^2_k
\end{equation}

% \url{https://risalc.info/src/st-chi-squared-distribution-summary.html}


\subsection{分割表における独立性とは}

下記のような分割表を考える。

\begin{table}[h]
  \centering
  \begin{tabular}{c|ll|c}
   &  $A_1$ & $A_2$ & Total \\ \hline
 $B_1$  & $y_{11}$ & $y_{12}$  & $y_{1\cdot}$ \\
 $B_2$  & $y_{21}$ & $y_{22}$  & $y_{2\cdot}$ \\ \hline
 Total  & $y_{\cdot 1}$ & $y_{\cdot 2}$ & $y_{\cdot\cdot}$ 
  \end{tabular}
\end{table}

各カテゴリの事象が観測される確率は
\begin{table}[h]
  \centering
  \begin{tabular}{c|ll|c}
   &  $A_1$ & $A_2$ & Total \\ \hline
 $B_1$  & $p_{11}$ & $p_{12}$  & $p_{1\cdot}$ \\
 $B_2$  & $p_{21}$ & $p_{22}$  & $p_{2\cdot}$ \\ \hline
 Total  & $p_{\cdot 1}$ & $p_{\cdot 2}$ & $p_{\cdot\cdot}$ 
  \end{tabular}
\end{table}
で表される。
条件付き確率で書き下すと、例えばセル$(1,1)$の生起確率は
\begin{equation}
  p_{11} = P(A_1|B_1)
\end{equation}
で表される。 属性$A$と$B$に関係性がないとき、
\begin{equation}
  p_{11} = P(A_1|B_1) = P(A_1)P(B_1)
\end{equation}
と表される。分割表において独立とは、属性間に関係性がないことを指す。


\subsection{適合度検定}

ある母集団から$n$個の標本が得られ、それらが$k$個のクラスに分類されているとする。
各クラスの観測数が$O_i$、理論上期待される値が$E_i$とするとき

\begin{equation}
  \sum_{i=1}^k \frac{(O_i-E_i)^2}{E_i}
\end{equation}

となる検定量を用いることがあり、これをピアソンのカイ二乗検定統計量と呼ぶ。
観測データの確率分布が理論上想定している確率分布に等しいとする帰無仮説のもとで、$n\to\infty$の場合に(=近似的に)自由度$k-1$のカイ二乗分布に分布収束する。

%- \url{file:///Users/ktakeda/Downloads/13460226_v61_3_p123-131_%E4%B8%AD%E5%B6%8B.pdf}


\begin{thebibliography}{99}
  \bibitem{hirotsu}  順序カテゴリの分割表における準対称モデルについて, 広津 \\
  \bibitem{ref2} xxx 
\end{thebibliography}

\end{document}
